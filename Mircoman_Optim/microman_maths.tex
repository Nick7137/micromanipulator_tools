\documentclass[11pt]{article}

\usepackage{amsmath,amssymb,amsthm,graphicx}
\usepackage[margin=1in]{geometry}
\usepackage{bm}

\begin{document}
\thispagestyle{empty}
\begin{center}
{\LARGE \textbf{Proof: angles for rotating disc and moving grabber}}\\[1em]
{\large (coordinate system: origin at top-left, $x$ rightwards, $y$ downwards)}
\end{center}

\bigskip
\noindent\textbf{Problem setup.}
We have a circular disc of radius $R$ with centre at the point $\mathbf{D}=(D_x,D_y)$ (Cartesian coordinates, origin at top-left, $x$ increasing to the right, $y$ increasing downwards). A point on the disc is $\mathbf{A}=(A_x,A_y)$. Independently there is a circular ``grabber'' of fixed radius $r$ whose centre is at $\mathbf{G}=(G_x,G_y)$. The grabber centre lies due south of the disc centre and the northern-most point of the grabber circle coincides with the disc centre; hence

\[
\mathbf{G}=\mathbf{D}+(0,r)\qquad\text{(i.e. }G_x=D_x,\;G_y=D_y+r\text{).}
\]

The grabber can move along its circular arc; we measure the grabber angular position $\theta$ as the clockwise angle from the NS line (the vertical line through $\mathbf{G}$ and $\mathbf{D}$), so that $\theta=0$ corresponds to the northernmost point of the grabber circle (which is the disc centre). The disc may be rotated by an angle $\alpha$ (clockwise positive) about its centre $\mathbf{D}$.

We seek expressions for (i) the rotation angle $\alpha$ that places the given point on the disc onto the grabber circle, and (ii) the corresponding grabber angle $\theta$ at which the grabber meets the (now rotated) point.

\bigskip

\noindent\textbf{Notation.}
Let
\begin{align*}
\Delta x&=A_x-D_x,\\
\Delta y&=A_y-D_y,\\
\mathbf{u}&=\mathbf{A}-\mathbf{D}=(\Delta x,\Delta y),\\
L&=|\mathbf{u}|=\sqrt{\Delta x^2+\Delta y^2}\quad(\text{distance from }\mathbf{D}\text{ to }\mathbf{A}).
\end{align*}
Define the angle $\beta$ of the vector $\mathbf{u}$ measured clockwise from the positive $x$-axis (this choice is consistent with the screen coordinates where $y$ increases downward):

\[
\beta=\operatorname{atan2}(\Delta y,\Delta x).
\]

Finally set

\[
S:=\frac{L}{2r}.\qquad(\text{note: a necessary condition for a solution is }L\le 2r\text{, i.e. }S\le 1.)
\]

\bigskip

\noindent\textbf{Step 1: equation for the rotated point lying on the grabber circle.}
Translate coordinates so the disc centre is at the origin: working with vectors relative to $\mathbf{D}$ is algebraically convenient. The vector from $\mathbf{D}$ to $\mathbf{A}$ is $\mathbf{u}=L(\cos\beta,\sin\beta)$ (where cosine and sine follow the same orientation convention as $\beta$). After rotating the disc clockwise by $\alpha$, the image of $\mathbf{A}$ relative to the disc centre is

\[
\mathbf{u}'=L\big(\cos(\beta+\alpha),\;\sin(\beta+\alpha)\big).
\]

The position of the rotated point relative to the grabber centre $\mathbf{G}$ is

\[
\mathbf{u}'-(0,r),\qquad\text{because }\mathbf{G}=\mathbf{D}+(0,r).
\]

Requiring that the rotated point lies on the grabber circle of radius $r$ centred at $\mathbf{G}$ yields

\[
\big\|\mathbf{u}'-(0,r)\big\|=r.
\]

Square both sides and expand:
\begin{align*}
\big|\mathbf{u}'-(0,r)\big|^2&=\big|\mathbf{u}'\big|^2 + r^2 -2r\big(\text{the $y$-component of }\mathbf{u}'\big)\\
&=L^2 + r^2 -2r\big(L\sin(\beta+\alpha)\big) = r^2.
\end{align*}
Cancel $r^2$ and solve for $\sin(\beta+\alpha)$:

\[
L^2-2rL\sin(\beta+\alpha)=0\quad\Longrightarrow\quad \sin(\beta+\alpha)=\frac{L}{2r}=S.
\]

\bigskip

\noindent\textbf{Step 2: solutions for $\alpha$.}
The identity $\sin(\beta+\alpha)=S$ gives the two (principal) solutions for $\beta+\alpha$ modulo $2\pi$:

\[
\beta+\alpha=\arcsin(S)\quad\text{or}\quad\beta+\alpha=\pi-\arcsin(S)\;\;\; (\bmod\;2\pi).
\]

Therefore the corresponding solutions for the disc rotation angle $\alpha$ (clockwise positive) are
\begin{equation}\label{eq:alpha}
\boxed{\qquad\alpha=\arcsin(S)-\beta\quad\text{or}\quad\alpha=\pi-\arcsin(S)-\beta\qquad}
\end{equation}
(Each value is understood modulo $2\pi$ and chosen according to which intersection on the grabber circle is required.)

\bigskip

\noindent\textbf{Step 3: relationship giving the grabber angle $\theta$.}
Let $\theta$ denote the clockwise angle along the grabber circle measured from the NS line (vertical through $\mathbf{G}$ and $\mathbf{D}$), with $\theta=0$ at the northernmost point (which equals $\mathbf{D}$). The parametric coordinates of a point on the grabber circle with angle $\theta$ are (in the same screen-coordinate convention)

\[
\mathbf{P}(\theta)=\mathbf{G}+r\big(\sin\theta,\,-\cos\theta\big).
\]

The rotated point we found also equals $\mathbf{P}(\theta)$; comparing the $y$-components of the equality $\mathbf{u}'-(0,r)=r(\sin\theta,-\cos\theta)$ gives

\[
L\sin(\beta+\alpha)-r = -r\cos\theta\quad\Longrightarrow\quad L\sin(\beta+\alpha)=r(1-\cos\theta).
\]

Using $\sin(\beta+\alpha)=S$, we get

\[
L\cdot S = r(1-\cos\theta)\quad\Longrightarrow\quad 1-\cos\theta=\frac{L^2}{2r^2}.
\]

Hence

\[
\cos\theta = 1-\frac{L^2}{2r^2} = 1-2S^2.
\]

Write this in half-angle form: recall $1-\cos\theta = 2\sin^2(\theta/2)$. Therefore

\[
2\sin^2\left(\frac{\theta}{2}\right)=\frac{L^2}{2r^2}=4S^2\cdot\frac{1}{4}=2S^2\quad\Longrightarrow\quad \sin^2\left(\frac{\theta}{2}\right)=S^2.
\]

Thus

\[
\frac{\theta}{2}=\arcsin(S)\quad\text{or}\quad\frac{\theta}{2}=\pi-\arcsin(S)\;\;\; (\bmod\;2\pi).
\]

Consequently the grabber angular solutions are
\begin{equation}\label{eq:theta}
\boxed{\qquad\theta=2\arcsin(S)\quad\text{or}\quad\theta=2\pi-2\arcsin(S)\qquad}
\end{equation}
(Again $\theta$ is understood modulo $2\pi$; take the branch in $[0,2\pi)$ appropriate to the chosen intersection.)

\bigskip

\noindent\textbf{Remarks and conditions.}
\begin{itemize}
\item The quantity $S=\dfrac{L}{2r}$ must satisfy $0\le S\le1$ for real solutions; therefore a necessary and sufficient condition for an intersection is $L\le 2r$. Geometrically this says the point on the disc (after rotation) must be at most diameter distance apart from the grabber centre so that the grabber circle can reach it.
\item The disc radius $R$ does not appear in the algebraic expressions for $\alpha$ and $\theta$ except insofar as the given point $\mathbf{A}$ must lie on the disc (so that $L\le R$). The grabber centre $\mathbf{G}$ is used only via the relation $\mathbf{G}=\mathbf{D}+(0,r)$ (the northernmost point of the grabber circle is at $\mathbf{D}$) and the grabber radius $r$ appears explicitly.
\item The two algebraic branches for $\alpha$ correspond to the two different intersection points on the grabber circle (the one on the left-hand side of the vertical through the centres and the one on the right-hand side). Similarly the two values of $\theta$ denote the clockwise and counter-clockwise arc positions that are symmetric with respect to the NS line.
\end{itemize}

\bigskip

\noindent\textbf{Final boxed formulas.}

\[
\boxed{\displaystyle\alpha = \arcsin\!\left(\frac{L}{2r}\right)-\beta\quad\text{or}\quad \alpha = \pi-\arcsin\!\left(\frac{L}{2r}\right)-\beta}
\]
\[
\boxed{\displaystyle\theta = 2\arcsin\!\left(\frac{L}{2r}\right)\quad\text{or}\quad\theta=2\pi-2\arcsin\!\left(\frac{L}{2r}\right)}
\]

where $L=|\mathbf{A}-\mathbf{D}|$ and $\beta=\operatorname{atan2}(\Delta y,\Delta x)$ (clockwise from the positive $x$-axis).  \qed

\bigskip

\noindent\textbf{(Optional) short coordinate check.}
If one places the disc centre at the origin and chooses a concrete $\mathbf{u}$, the algebra above reduces to the elementary trigonometric equalities used in the derivation; the two branches correspond to the two solutions of the sine and cosine equations.

\end{document}
